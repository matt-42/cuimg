\documentclass{beamer}


\mode<presentation> {
%  \usetheme{Warsaw}
  \usetheme{Frankfurt}
  % ou autre ...

%\setbeamertemplate{footline}[frame number]
%  \setbeamertemplate{navigation symbols}{}


%\setbeamertemplate
%{footline}
%{\quad\strut\insertsection
%\hfill\insertframenumber/\inserttotalframenumber\strut\quad}

%  \useinnertheme[shadow=false]{rounded}

  \setbeamercovered{transparent}
\useoutertheme[subsection=false, footline=authortitle]{miniframes}

  % ou autre chose (il est également possible de supprimer cette ligne)
}

\newcommand*\oldmacro{}%
\let\oldmacro\insertshorttitle%
\renewcommand*\insertshorttitle{%
  \oldmacro\hfill%
  \insertframenumber\,/\,\inserttotalframenumber}


\usepackage[utf8]{inputenc}
\usepackage[T1]{fontenc}

%\usepackage[usenames]{color}
%\usepackage[usenames,dvipsnames]{color}
%%%%%\usepackage{subfigure}

% Anti aliasing with pdflatex
\usepackage{aeguill}

%\usepackage{fullpage}
%\usepackage[pdfborder={0 0 0}, bookmarks,
%  bookmarksopen, bookmarksnumbered,
%  bookmarksopenlevel={2}, backref,pdfauthor={},
%  pdftitle={}]{hyperref}

\usepackage{amsmath,amsfonts,amscd,amssymb}
\usepackage{listings,textcomp}
\usepackage{graphicx}
%\usepackage{afterpage}
%\usepackage{float}
%\usepackage{multicol}
\usepackage{ifthen}
\usepackage{tikz}
\usetikzlibrary{calc}
\usetikzlibrary{plotmarks}
\usetikzlibrary{positioning}
\usetikzlibrary{shapes}
\usetikzlibrary{fit}

\pgfdeclarelayer{background}
\pgfdeclarelayer{foreground}
\pgfsetlayers{background,main,foreground}

\usepackage{pgfplots}
\usepackage{array}
\usepackage{multicol}
\usepackage{verbatim}

%% info sur le document
\title{Cuimg: A small image processing framework based on CUDA}
\author[Matthieu Garrigues]{Matthieu Garrigues  \texttt{<matthieu.garrigues@ensta.fr>}}

\date{\today}

\AtBeginSubsection[] {
  \begin{frame}<beamer>{Overview of the presentation}
    \tableofcontents[currentsection,currentsubsection]
  \end{frame}
}

\newcommand \pgfmathparseint[2]
{
  \pgfmathparse{#2};
  \pgfmathfloatparsenumber{\pgfmathresult}
  \pgfmathfloattoint{\pgfmathresult};
  \let#1\pgfmathresult;
}

\DeclareMathOperator*{\argmax}{arg\,max}

\usepackage[scaled]{beramono}

\lstset{
  language=C++,
  keywordstyle=\bfseries\ttfamily\color[rgb]{0,0.3,0},
  identifierstyle=\ttfamily,
  commentstyle=\color[rgb]{0.133,0.545,0.133},
  stringstyle=\ttfamily\color[rgb]{0.627,0.126,0.941},
  showstringspaces=false,
  basicstyle=\tiny,
%  numberstyle=\footnotesize,
  numbers=left,
  stepnumber=1,
  numbersep=10pt,
  tabsize=2,
  breaklines=true,
  breakatwhitespace=false,
%  aboveskip={1.5baselineskip},
  columns=fixed,
  upquote=true,
  extendedchars=true,
  mathescape=true,
  morecomment=[l]{//},
  emph={selectmin, selectmax, reorder},
  emphstyle=\bfseries\ttfamily\color{MidnightBlue},
  escapeinside={(*@}{@*)}
}

\begin{document}

\begin{frame}
\maketitle
\end{frame}


\begin{frame}{Outline}
  \tableofcontents
\end{frame}

\section{Goals}

\begin{frame}{Goals}
  \begin{itemize}
  \item Write efficient image processing applications
  \item With CUDA
  \item Reduce the size of the application code
  \item ... and the number of potential bugs
  \end{itemize}
\end{frame}

\section{Bases}

\begin{frame}[containsverbatim]{Image types}
\begin{lstlisting}
// Host types.
template <typename T> host_image2d;
template <typename T> host_image3d;
// CUDA types.
template <typename T> image2d;
template <typename T> image3d;
// Types used inside CUDA kernels.
template <typename T> kernel_image2d;
template <typename T> kernel_image3d;

{
  device_image2d<float4> img(100, 100); // creation.
  device_image2d<float4> img2 = img; // light copy.
} // Images are freed automatically here.

// See a slice as a 2d image:
device_image3d<float4> img3d(100, 100, 100);
device_image2d<float4> slice = img3d.slice(42);
\end{lstlisting}
\end{frame}

\begin{frame}[containsverbatim]{Improved builtins types}
\begin{lstlisting}
i_float3 a(1.5, 1.5, 1.5);
i_float3 b(2.5, 2.5, 2.5);

// Classical arithmetic operators
i_float3 c = (a + b) * 2.5f;

// Interoperability
i_char2 x = i_int2(0,1) + i_float2(0,1) * 5.0f;
// Type of (i_int2 + i_float2) is i_float2 as
// (int + float) returns a float.
\end{lstlisting}
\end{frame}

\begin{frame}[containsverbatim]{Inputs}
\begin{lstlisting}
// Load 2d images.
host_image2d<uchar3> img = load("test.jpg");

// Read USB camera
video_capture cam(0);
host_image2d<uchar3> cam_img(cam.nrows(), cam.ncols());
cam >> cam_img; // Get the camera current frame

// Read a video
video_capture vid("test.avi");
host_image2d<uchar3> frame(vid.nrows(), vid.ncols());
vid >> v; // Get the next video frame
\end{lstlisting}
\end{frame}

\begin{frame}[containsverbatim]{Data tranfers}
\begin{lstlisting}

host_image2d<uchar3> img_h = load("test.jpg"); // Lives in CPU RAM
device_image2d<uchar3> img(img_h.domain()); // Lives in GPU RAM

copy(img_h, img); // CPU -> GPU
copy(img, img_h); // GPU -> CPU

// Also works on 3d images.
\end{lstlisting}
\end{frame}


\section{Algorithms}

\begin{frame}[containsverbatim]{Fast gaussian convolutions code generation}

\begin{columns}
\begin{column}[l]{5cm}
  \begin{itemize}
  \item Heavy use of C++ templates for loop unrolling
  \item Gaussian kernel is known at compile time and injected directly
    inside the assembly
  \item Used by the local jet computation
  \item Cons: Large kernels slow down the compilation.
  \end{itemize}
\end{column}
\begin{column}[r]{5cm}
\tiny{mov.f32 	\%f182, \textbf{0f3b1138f8;   	// 0.00221592} \\
mul.f32 	\%f183, \%f169, \%f182;\\
mov.f32 	\%f184, \textbf{0f39e4c4b3;   	// 0.000436341} \\
mad.f32 	\%f185, \%f184, \%f181, \%f183;\\
mov.f32 	\%f186, \textbf{0f3c0f9782;   	// 0.00876415} \\
mad.f32 	\%f187, \%f186, \%f157, \%f185;\\
mov.f32 	\%f188, \textbf{0f3cdd25ab;   	// 0.0269955} \\
mad.f32 	\%f189, \%f188, \%f145, \%f187;\\
mov.f32 	\%f190, \textbf{0f3d84a043;   	// 0.0647588} \\
mad.f32 	\%f191, \%f190, \%f133, \%f189;\\
mov.f32 	\%f192, \textbf{0f3df7c6fc;   	// 0.120985} \\
mad.f32 	\%f193, \%f192, \%f121, \%f191;\\
mov.f32 	\%f194, \textbf{0f3e3441ff;   	// 0.176033} \\
mad.f32 	\%f195, \%f194, \%f109, \%f193;\\
mov.f32 	\%f196, \textbf{0f3e4c4220;   	// 0.199471} \\
mad.f32 	\%f197, \%f196, \%f97, \%f195;\\
mov.f32 	\%f198, \textbf{0f3e3441ff;   	// 0.176033} \\
mad.f32 	\%f199, \%f198, \%f85, \%f197;\\
mov.f32 	\%f200, \textbf{0f3df7c6fc;   	// 0.120985} \\
mad.f32 	\%f201, \%f200, \%f73, \%f199;\\
mov.f32 	\%f202, \textbf{0f3d84a043;   	// 0.0647588} \\
mad.f32 	\%f203, \%f202, \%f61, \%f201;\\
mov.f32 	\%f204, \textbf{0f3cdd25ab;   	// 0.0269955} \\
mad.f32 	\%f205, \%f204, \%f49, \%f203;\\
mov.f32 	\%f206, \textbf{0f3c0f9782;   	// 0.00876415} \\
mad.f32 	\%f207, \%f206, \%f37, \%f205;\\
mov.f32 	\%f208, \textbf{0f3b1138f8;   	// 0.00221592} \\
mad.f32 	\%f209, \%f208, \%f25, \%f207;\\
mov.f32 	\%f210, \textbf{0f39e4c4b3;   	// 0.000436341}}
\end{column}
\end{columns}

\end{frame}

\begin{frame}[containsverbatim]{A simple CUDA kernel}

\begin{lstlisting}

// Declaration
template <typename T, typename U, typename V>
__global__ void simple_kernel(kernel_image2d<T> a,
                              kernel_image2d<U> b,
                              kernel_image2d<V> out)
{
  point2d<int> p = thread_pos2d();
  if (out.has(p))
    out(p) = a(p) * b(p) / 2.5f;
}

// Call
dim3 dimblock(16, 16);
dim3 dimgrid(divup(a.ncols(),  16), divup(a.nrows(),  16));
simple_kernel<<<dimgrid, dimblock>>>(a, b, c);
\end{lstlisting}

\end{frame}

\begin{frame}[containsverbatim]{Simple kernels generator}

  \begin{block}{Goal}
  Automatically generate simple kernels
  \end{block}

\begin{block}{Examples}
\begin{lstlisting}
device_image2d<i_int4> a(200, 100);
device_image2d<i_short4> b(a.domain());
device_image2d<i_float4>   c(a.domain());

// Generates and calls simple_kernel.
c = a * b / 2.5;

// BGR <-> RGB conversion
a = aggregate<int>::run(get_z(a), get_y(a), get_x(a), 1.0f);
\end{lstlisting}
\end{block}
\end{frame}

\begin{frame}[containsverbatim]{Simple kernels generator}

    \begin{lstlisting}
  // C++ templates handle the syntax tree of the expression
  // Type of
  a * b / 2.5
  // is
  div<mult<device_image2d<i_int4>,
           device_image2d<i_short4> >,
      float>


  //operator= handles the kernel call
  template <typename I, typename E>
  __global__ void assign_kernel(kernel_image2d<I> out, E e)
  {
    i_int2 p = thread_pos2d();
    if (out.has(p))
      out(p) = e.eval(p);
      // Recursivelly evaluates the expression, inlining of the recursive
         evaluation of the tree is done by the compiler.
  }

  template <typename A, template <class> class AP, typename E>
  inline void
  image2d::operator=(E& expr)
  {
    assign_kernel<<<dimgrid, dimblock>>>(*this, e);
  }

    \end{lstlisting}

\end{frame}

\begin{frame}

 Question?

\end{frame}

\end{document}
